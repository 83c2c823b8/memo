\RequirePackage{luatex85}
\documentclass[leqno]{ltjsarticle}% leqnoは数式番号左
\usepackage{luatexja-fontspec}
\usepackage[top=10truemm,bottom=10truemm,left=20truemm,right=20truemm]{geometry}
\usepackage{luatexja} 
\usepackage{multicol,amsmath,amssymb,mathtools,ascmac,amsthm,amscd,physics,comment,dcolumn,titlesec,mathrsfs,mypkg}
\usepackage[all]{xy}
\titleformat*{\section}{\Large\bfseries}
\setlength{\parindent}{0pt}
\pagestyle{empty}
%\everymath{\displaystyle}
\begin{document}
{\textbf{\Large{院試}}}\hspace{\fill} {理学部数学科\texttt{\Large{04A21051}}}{\Large{山本 雄大}}\\
距離空間で考える.
\mythm{
	コンパクト集合上の連続関数は最大値,最小値を持つ.
}
$K$コンパクト,$f$連続とする.
\[R \coloneq f(K)\]
$b\coloneq \sup R$として,$R$の点列コンパクト性を示せばよい. \\
点列$\{y_n\}_n$を任意にとる.$f(x_n)=y_n$となる$\{x_n\}_n$が存在して,$K$点列コンパクトより$x_{n(k)}\to a (a\in K)$となる収束部分列が存在する.
このとき,$y_{n(k)}\to f(a) \in R$したがって,点列コンパクトである.
\mythm{
	コンパクト集合上の連続関数は一様連続
}
コンパクト集合$K$と連続関数$f$を考え,$\varepsilon > 0$を任意に固定する.\\
$f$連続より,各点$x\in K$においてある$\delta_{x}$が存在して,$d(x,y) < \delta_{x} \Longrightarrow |f(x) - f(y) | < \frac12\varepsilon $\\
\[U_x = \{y \in K \mid d(x,y) < \frac12 \delta_x\}\] 
(ここの$1/2 \delta_x$ のとり方がカシコイ)\\
と定めれば$U_x$は$K$の開被覆であり,コンパクトであることから有限個におとせる.
\[\bigcup_{i=1}^n U_{x_i} \supseteq K\]
$\delta \coloneq \frac12 \min(\delta_{x_1},\ldots , \delta_{x_n})$とさだめると\\
$d(x,y)< \delta$をみたす任意の$x,y \in K $に対して,ある$i$が存在して,$x\in U_{x_i}$.したがって,$d(x,x_i) < \frac12 \delta_{x_i}$\\
連続性より
\[|f(x_i) - f(x) |< \frac12 \varepsilon\]
また,
\[d(y,x_i) \leq d(y,x) + d(x,x_i) < \delta + \frac12 \delta_{x_i} < \delta_{x_i}\]
\[|f(x_i) - f(y)| < \frac12 \varepsilon\]
したがって
\[|f(x) - f(y)|<\varepsilon \]
\mythm{
	コンパクト$\Longleftrightarrow$ 点列コンパクト
}

\mythm{
	$|H| = |xHx\inv|$
}
$h\mapsto xhx\inv$は全単射\\
全射は定義より,単射は$xhx\inv  = xh'x\inv$なら$h=h'$

\mydef{
	$\sigma' = \tau\sigma\tau\inv\ \  (\exists \tau\in G)$
	のとき$\sigma'$と$\sigma$は共役.
}	

\mydef{
	対称群のサイクルタイプとは
	\[\sigma = (i_1\cdots i_s)(j_1 \cdots j_t)(k_1 \cdots k_u) \]
	の$(s,t,\ldots ,u)$
}

\mythm{
	$S_n$で共役$\Longleftrightarrow$サイクルタイプが一致
}

\mythm{
	$S_n$の共役類数は分割数$p(n)$
}

\mythm{
		$x$の共役類を$C(x)$,$G$の中心を$Z$と記すと
		$|C(x)|,|Z|$は$|G|$を割り切る.
}
\[g\circ x \coloneq gxg\inv\]
と作用を定義すると,作用による軌道の要素の数と固定する部分群の位数の積が$G$の位数になる.\\
$x$の$G$における固定部分群を$\mathrm{Stab_G}(x)$としるすと,全射$f\colon G \to \mathrm{Orb_G}(x)$
の核は$\mathrm{Stab_G}(x)$なので順同型定理っぽくやればわかる.

\mythm{
	$p$群の中心は非自明
}
$|Z|=1$とすると類等式から矛盾が生じる.

\mythm{
	$p$群の位数$p$の部分群$H$が正規部分群$\Longleftrightarrow$ $H\subseteq Z$
}
$\{1\}$が一つの軌道になっていることと,軌道が$|G|=p^n$を割り切ること$|H|=p$であることから
各点が一つの軌道をなしていることからしたがう.

\mythm{
	$p$が$|G|$の最小の素因数で$H<G,[G:H]=p$なら$H$は正規部分群である.
}
\[\phi\colon G \to S(G/H) \simeq \mathfrak{S}_p\]
の作用を考えると$\ker \phi \subseteq H$
\[[G:\ker \phi] = [G:H][H:\ker\phi] = p[H:\ker\phi] \]
ここで,$[G:\ker\phi]=|G/\ker\phi|=|\Im\phi|$であり,$\Im\phi$は$\mathfrak{S}_n$の部分群なので$|\Im\phi|$ は$|\mathfrak{S}_n|=p!$を割り切る.\\
$[G:\Ker\phi]=p$でなければ,$[G:\ker\phi]$が$p-1$以下の約数を持つことになり仮定に矛盾.
したがって,$[H:\ker\phi]=1$,核は正規部分群なのでok.

\mythm{
	$G$が$H,K$の直積になる.\vspace{-3mm}
			\begin{enumerate}
				\item[(1)]
					$H,K$は$G$の正規部分群
				\item[(2)]
					$H\cap K = 1$
				\item[(3)]
					$G=HK$
			\end{enumerate}
}

\mythm{
	$\{a_n\}_n$は$a_n\leq 0$かつ広義単調減少,$\lim_{n\to\infty}a_n = 0$
	\[\sum_{n=0}^\infty (-1)^{n+1}a_n\]は収束.
}
二項ずつ考えればできる.

\mythm{
	d'Alembertの収束判定法\\
	$\sum a_n$について
	\[\limsup_{n\to\infty}\qty|\frac{a_{n+1}}{a_n}| < 1\]
	なら収束
	\[\liminf_{n\to\infty}\qty|\frac{a_{n+1}}{a_n}| > 1\]
	なら発散
}
収束する方は,ある$N$より大きいとその絶対値の級数は$(1-\varepsilon)$の等比数列で抑えられる.
発散する方は$\lim_{n\to\infty}a_n$が0に収束しない

\mythm{
	Cauchyの収束判定法\\
	$\sum a_n$について
	\[\limsup_{n\to\infty}\sqrt[n]{|a_n|}=r\]
	$0\le r < 1$なら絶対収束
	$r>1$なら発散
}
ダランベールと同じ感じ
\end{document}
